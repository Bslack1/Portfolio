\documentclass{beamer}
\usepackage{xcolor}
\usepackage{amsmath}
\usefonttheme[onlymath]{serif}
\begin{document}

\begin{frame}

\begin{center}

\Huge \color{orange}\textbf{Numerical Analysis:} 

\Large \color{purple}Important Methods \\

\bigskip

\large \color{cyan}Bernadette Hoffman \\

\normalsize \color{black} December $13^{th}$, 2021

\end{center}

\end{frame}

\begin{frame}

\frametitle{\color{orange}Newton's Method}

\normalsize \color{black}
When approximating the root of a function Newton's method is one of the most commonly used because of its ease of use.\\ Proof:
\begin{center}
\color{purple} 
\begin{align*}
 f'(x_n) &= \frac{f(x_n)}{x_n - x_{n+1}} \\ 
 f'(x_n)(x_n - x_{n+1}) &= f(x_n) \\ 
 x_n - x_{n+1} &= \frac{f(x_n)}{f'(x_n)} \\ 
 -x_{n+1} &= \frac{f(x_n)}{f'(x_n)} - x_n \\ 
 x_{n+1} &= x_n - \frac{f(x_n)}{f'(x_n)} \\ 
\end{align*} 
\color{black}The final product gives Newton's method for solving $f(x)=0$.
\end{center}
\end{frame}

\begin{frame}
\frametitle{\color{orange}Secant Method}
Where Newton's method uses the tangent line to approximate the root of a function, the Secant method uses (obviously) the secant line. \\
\color{purple}
$$ \frac{f(x_n)-0}{x_n - x_{n+1}} = \frac{f(x_{n-1} - f(x_n)}{x_{n-1} - x_n} $$ \pause
\color{black}Becomes: \\ \pause
\color{purple}$$ x_{n+1} = x_n - f(x_n) \cdot \frac{x_n - x_{n-1}}{f(x_n) - f(x_{n-1})} $$

\end{frame}

\begin{frame}
\frametitle{\color{orange}Bisection Method}
The bisection method is more of a computing algorithm than a formula. Also used for approximating the root of a function, the Bisection method works as follows:
\vspace{1cm}
\color{purple}
\begin{center}
\begin{enumerate}
\item Define $c = \frac{(a+b)}{2}$ \pause
\item If $b - c \le \epsilon \implies c$ is the root. Stop.\pause
\item If $\pm [f(b)] \cdot \pm [f(c)] \le 0 \implies a=c$ \\ If $\pm \ge 0 \implies b = c$.
\end{enumerate}
\end{center}
\end{frame}













\end{document}
