\documentclass[oneside]{book}
\usepackage{amsmath}
\usepackage{MnSymbol}
\usepackage{xcolor}
\usepackage{fullpage}
\usepackage{graphicx}
\usepackage{wrapfig}
\usepackage[skins]{tcolorbox}
\usepackage{tocbibind}
\usepackage[utf8]{inputenc}
\usepackage[backend=biber,
            style=alphabetic,
            sorting=ynt]{biblatex}
            
\addbibresource{biblioFinal.bib}

\begin{document}
\definecolor{beige}{RGB}{217, 217, 132}
\definecolor{mypink}{RGB}{235, 63, 86}
\definecolor{myblue}{RGB}{5, 150, 247}
\frontmatter %starts my front section
\begin{titlepage}
\begin{center}
\vspace*{3in}
\textbf{\Huge \color{purple}42\\ \huge The Answer to Life, \\the Universe, and Everything}

\bigskip

\LARGE{\color{black}By Bernadette Hoffman}
\bigskip
\bigskip
\large \\Math 150 - Intro to \LaTeX \\ Final \\ December $4^{th}$, 2021
\end{center}
\end{titlepage}

\tableofcontents
\thispagestyle{empty}
\cleardoublepage

\mainmatter %switch to body
\setcounter{page}{1}
\chapter{Autobiography}\label{sec:bio}
\section{Early Education}

Born in 1984, I was born into a fundamentalist Christian family in a small town on the eastern shore of Virginia. In 2nd grade I was skipped ahead to 3rd grade because of my math comprehension. The next year my parents decided to withdraw me and my sister from public school. This was done because of the religious belief that women are incapable of gaining secular knowledge and staying faithful to god. A new belief held by the organization my parents followed. 

My continued education centered around the bible and keeping up with the states standardized testing requirements for home schooled children (which in the mid nineties was not rigid at all). Unfortunately my parents did not continue with any math or science education. My math classes were completely halted from 3rd grade until my freshman year of high school when I was presented with Algebra 1. Similarly, my entire high school education consisted of one creation-friendly biology class. I moved to Florida in December of 1999.

I graduated at the age of 17 from the American School of Illinois. Upon graduating I received a scholarship geared towards home-schooled children from the state of Florida to attend the school of my choice, never having to complete an ACT or SAT. My parents allowed me to attend a private or trade school of my choice. Since I lived in rural north central Florida, there was only one choice. I enrolled at Webster College into the Information Technology program. Despite my severe lack of education I picked things up pretty well. After the completion of my first year Webster college lost their accreditation. I looked into transferring schools and was abruptly discouraged realizing that I would have to take an ACT or SAT test in addition to having to start all over again. My college days were over. 

\section{Career Choices}

Once I resigned myself to never obtaining a degree, I obtained my mortgage brokers license in the state of Florida. I was hired by SunTrust Bank and quickly progressed from teller to loan officer. I moved to West Virginia in 2007. After several drastic life changes, I realized that I wanted to go back to school. I knew that the task ahead was daunting, since I still had huge gaps in my education. I bought all of the "for dummies" algebra, and geometry books I could find and got to work. At this point I did not make it back to school, and eventually ended up in management at SunTrust Bank. 

In 2018 SunTrust Bank announced that they planned to shut down all branches in West Virginia. I was given the option of a severance or transferring out of state. It dawned on me, at 34 years old, that this was my last chance to obtain the education I always new I wanted. As people left the area, I was one of the last two employees at my branch. I spent around 2 months working open to close 6 days a week so that the branch would remain open. I spent most of each work day studying so that I would be ready to apply to schools. 

\section{College bound ... again}

In 2018 I was accepted to West Virginia State University as a student of Computer Sciences. I was so proud of myself, while shocked that I actually obtained admittance to a real university. Realizing that there were major gaps in my education I demanded to be placed in intermediate algebra, despite being told that I didn't need to take that class. Class was at times a rude awakening. Simple things that people take for granted learning in a decent education were lost one me. I excelled in English classes but struggled in most classes related to my major or the natural sciences. 

In my second year I took Trigonometry. I struggled so much to keep up and understand what was going on. The basic knowledge of mathematical concepts required for the course, were completely foreign to me. I watched Khan academy videos until I was blue in the face. During one of the videos I was learning about the 30-60-90 triangle. I didn't believe the concept. I had to prove it to myself. So I found graph paper and began graphing the unit circle. I drew each major angle and the corresponding coordinate values. While doing this the most beautiful thing happened. Something clicked. It all finally made sense, I had never felt such exhilaration in my life. Zip-lining in Costa Rica paled in comparison to that feeling. 

\section{A love of math develops}
I always loved doing math when I was in school. On some level I always knew that I had some (albeit minor) propensity for grasping mathematical concepts. Unfortunately, when I expressed my desire to go into programming and web design as a young child I was told "you can't do that, you're bad at math". I believed that I was bad at math until that sudden momentous epiphany trying to make sense of the unit circle. I may not be "good" at math in the classic sense. I still find that I struggle to understand certain concepts. Sometimes when I'm stuck I have to start digging to discover what basic concept has fallen through the gaps in my education. Math homework always becomes my priority because of the enjoyment from simply solving puzzles. I realized that semester that I enjoyed math more than programming. Eventually, I decided to update my enrollment to a dual major in Math and Computer Sciences. Since that day in Trigonometry I've discovered a great beauty in math. Though I will never claim to have any skill within the natural sciences, learning about places that math occurs in nature such as Fibonacci's sequence continues to fascinate me.

\cite{myBook}

\chapter{A Mathematician's Biography}

\begin{wrapfigure}{r}{.4\textwidth}
\centering
\includegraphics[width = .3\textwidth]{hypatia.jpg}
\end{wrapfigure}

Hypatia of Alexandria lived from 370 C.E. to March of 415 C.E., she was a female philosopher and mathematician. Raised by her father Theon who was also a mathematician, who saw to her education. Theon refused to force his daughter to fulfill the traditional roles for women at the time. Much to the contrary, he raised her as he would a son, educating her according to his profession and skills. There are accounts professing that her brilliance surpassed that of her father. She eventually became the first female respected academic at Alexandria University. Unfortunately, in a time when Christianity was gaining its foothold in society, Hypatia was a rationalist. This led to her being marred as a witch. A Christian mob captured and tortured her, burning what was left of her body. She was only 45. 

\bigskip

Though not much remains of Hypatia’s work or historical accounts, her students accounts remain for us today. Hypatia’s contributions to mathematics included work on algebraic equations and conic sections. She invented navigational devices for seafaring and tools for measuring the density of fluids. Hypatia was one of the first women in history to not only contribute to mathematics, but also to teach them. She was renowned as an excellent public speaker. More specifically Hypatia was known for making complicated mathematics easier to understand. Surviving accounts of her life left by former students sing her praise. Hypatia was truly a pathfinder for women in those times, as well as modern women in mathematics.\\

\cite{Hypatia}

\chapter{Calculus}

\section{\color{mypink}Fundamental Theorem of Calculus}
\color{black}
You have now been introduced to the two major branches of calculus: differential calculus (introduced with the tangent line problem) and integral calculus (introduced with with the area problem). At this point, these two problems might seem unrelated - but there is a very close connection. The connection was discovered independently by Isaac Newton and Gottfried Leibnix and is stated in a theorem that is appropriately called the \textbf{Fundamental Theorem of Calculus}.

Informally, the theorem states that differentiation and (definite) integration are inverse operations, in the same sense that division and multiplication are inverse operations. To see how Newton and Leibniz might have anticipated this relationship, consider the approximations shown in Figure 4.26. The slope of the tangent line was defined using the \textit{quotient} $\Delta$y/$\Delta$x (the slope of the secant line). Similarly, the area of a region under a curve was defined using the \textit{product} $\Delta$y$\Delta$x (the area of a rectangle). So, at least in the primitive approximation stage, the operations of differentiation and definite integration appear to have an inverse relationship in the same sense that division and multiplication are inverse operations. The Fundamental Theorem of Calculus states that the limit processes (used to define the derivative and definite integral) preserve this inverse relationship.

\bigskip 
 
%Figure out shadow, not showing
\begin{tcolorbox}[colback = white!5!white, 
				  colframe = myblue,
				  colbacktitle = white!5!white,
				  drop shadow southeast, 
				  enhanced,
				  sharp corners = all, 
title =\color{myblue}\textbf{THEOREM 4.9 \color{black} THE FUNDAMENTAL THEOREM OF CALCULUS}]
If a function $f$ is continuous on the closed interval $[a,b]$ and $F$ is an antiderivative of $f$ on the interval $[a,b]$, then

$$ \int_{a}^{b} f(x) \,dx \ = F(b) - F(a).$$ 

\end{tcolorbox}

\bigskip
%proof image or box
\color{myblue}
\noindent \textbf{(PROOF)} \color{black} \hspace{.2cm} The key to the proof is in writing the difference F(b) - F(a) in a convenient form. Let $\Delta$ be any partition of [$a, b$].

$$ a = x_0 < x_1 < x_2 < ... < x_{n-1} < x_n = b $$

\noindent By pairwise subtraction and addition of like terms, you can write

\begin{align*}
F(b) - F(a) &= F(x_n) - F(x_(n-1) - ... - F(x_1) + F(x_1) - F(x_0)
&= \sum_{i=1}^{n} [F(x_i) - F(x_{i-1})].
\end{align*}

\noindent By the Mean Value Theorem, you know that there exists a number $c_i$ in the $i$th subinterval such that

$$F'(c_i) = \frac{F(x_i) - F(x_{i-1})}{x_i - X_{i-1}}.$$

\noindent Because $F'(c_i) = f(c_i)$, you can let $\Delta x_i = x_i - x_{i-1}$ and obtain

$$ F(b)-F(a) = \sum_{i=1}^{n} f(c_i)\Delta x_i. $$

\noindent This important equation tells you that by repeatedly applying the Mean Value Theorem, you can always find a collection of $c_i$'s such that the \textit{constant} F(b)- F(a) is a Riemann sum of $f$ on $[a,b]$ for any partition. Theorem 4.4 guarantees that the limit of Riemann sums over the partition with $\| \Delta \| \to 0$ exists. So, taking the limit as (as $\| \Delta \| \to 0$) produces

$$ F(b) - F(a) = \int_{a}^{b} f(x) \,dx. $$

\bigskip

The following guidelines can help you understand the use of the Fundamental Theorem of Calculus.
\sffamily
\begin{tcolorbox}[colback = beige!75!white,
				  sharp corners = all,
				  colframe = beige!75!white]
\textbf{GUIDELINES FOR USING THE FUNDAMENTAL THEOREM OF CALCULUS}
\rmfamily
\begin{enumerate}
\item \textit{Provided you can find} an antiderivative of $f$, you now have a way to evaluate a definite integral without having to use the limit of a sum.

\item When applying the Fundamental Theorem of Calculus, the following notation is convenient
$$ \left. \int_{a}^{b} f(x) \,dx = F(x) \right]_a^b$$
$$ = F(b) - F(a) $$

For instance, to evaluate $ \int_{1}^{3} x^3 \,dx $, you can write
$$ \int_{1}^{3} x^3 \,dx = \frac{x^4}{4} = \frac{3^4}{4} - \frac{1^4}{4} = \frac{81}{4} - \frac{1}{4} = 20.$$

\item It is not necessary to include a constant of integration \textit{C} in the antiderivative because
\begin{align*}
\int_{a}^{b} f(x) \,dx &= \bigg[ F(x) + C \bigg]_a^b \\
  &= [F(b) + C] - [F(a) + C] \\
 &= F(b) - F(a) 
\end{align*}
\end{enumerate}
\end{tcolorbox}

\pagebreak

\section{\color{mypink}Integration by Parts}
\color{black} \rmfamily
In this section you will study an important integration technique called \textbf{integration by parts.} This technique can be applied to a wide variety of functions and is particularly useful for integrands involving \textit{products} of algebraic and  functions. For instance, integration by parts works well with integrals such as

$$ \int x\ ln\ x\ dx, \hspace{.2cm} \int x^2\ e^x\ dx, \hspace{.2cm} and \hspace{.2cm} \int e^x\ \sin x\ dx. $$

\noindent Integration by parts is based on the formula for the derivative of a product 

\begin{align*}
\frac{d}{dx} [uv] &= u \frac{dv}{dx} + v \frac{du}{dx}\\
 &= uv' + vu'
\end{align*}

\noindent where both \textit{u} and \textit{v} are differentiable functions of \textit{x}. If $u'$ and $v'$ are continuous, you can integrate both sides of this equation to obtain

\begin{align*}
uv &= \int uv' \ dx + \int vu' \ dx\\
&= \int u + \int v \ du.
\end{align*}

\noindent By rewriting this equation, you obtain the following theorem.

\begin{tcolorbox}[colback = white!5!white, 
				  colframe = myblue,
				  colbacktitle = white!5!white,
				  drop shadow southeast, 
				  enhanced,
				  sharp corners = all, 
title =\color{myblue}\textbf{THEOREM 8.1 \color{black} INTEGRATION BY PARTS}]
If $u$ and $v$ are functions of $x$ and have continuous derivatives, then

$$ \int u \ dv \ = \ uv \ - \int v \ du.$$
%need to align integral further left
\end{tcolorbox}

\bigskip

This formula expresses the original integral in terms of another integral. Depending on the choices of $u$ and $dv$, it may be easier to evaluate the second integral than the original one. Because the choices of $u$ and $dv$ are critical in the integration by parts process, the following guidelines are provided. 

\sffamily
\begin{tcolorbox}[colback = beige!75!white,
				  sharp corners = all,
				  colframe = beige!75!white]
\textbf{GUIDELINES FOR INTEGRATION BY PARTS}
\rmfamily
%Try to move indent left
\begin{enumerate}
\item[\textbf{1.}] Try letting $dv$ be the most complicated portion of the integrand that fits a basic integration rule. Then $u$ will be the remaining factor(s) of the integrand.

\item[\textbf{2.}] Try letting $u$ be the portion of the integrand whose derivative is a function simpler than $u$. Then $dv$ will be the remaining factor(s) of the integrand.
\end{enumerate}

\noindent Note that $dv$ always includes the $dx$ of the original integrand. 
\end{tcolorbox}

\bigskip

\noindent \color{myblue} \large \textbf{EXAMPLE} \color{black} \normalsize \textbf{Integration by Parts}
\rmfamily

\bigskip

\noindent Find $ \displaystyle \int xe^x \ dx$.
\sffamily

\bigskip

\noindent \color{myblue} \textbf{SOLUTION} \color{black} \rmfamily \hspace{.3cm}To apply integration by parts, you need to write the integral in the form $\int u \ dv$. There are several ways to do this.

$$ \int \color{mypink}\underbrace{\color{black}(x)}_\text{u} \color{mypink}\underbrace{\color{black}(e^x dx)}_\text{dv}\color{black}, 
\hspace{.2cm} \int \color{mypink}\underbrace{\color{black}(e^x)}_\text{u} \color{mypink}\underbrace{\color{black}(x dx)}_\text{dv}\color{black}, 
\hspace{.2cm}  \int \color{mypink}\underbrace{\color{black}(1)}_\text{u} \color{mypink}\underbrace{\color{black}(xe^x dx)}_\text{dv}\color{black}, 
\hspace{.2cm} \int \color{mypink}\underbrace{\color{black}(x e^x)}_\text{u} \color{mypink}\underbrace{\color{black}(dx)}_\text{dv} $$

\noindent The guidelines on page 527 suggest the first option because the derivative of $u = x$ is simpler than $x$, and $dv = e^x \ dx$ is the most complicated portion of the integrand that fits a basic integration formula. 


\begin{alignat*}{3}
dv = e^x \ dx \hspace{.5cm}  &\color{mypink}\Rightarrow \color{black}&v &= \int dv = \int e^x \ dx = e^x\\
u = x \hspace{.5cm} &\color{mypink}\Rightarrow \color{black}& \hspace{.3cm}du &= dx
\end{alignat*}

\noindent Now, integration by parts produces

\begin{align*}
\int u \ dv &= uv - \int v \ du \ \hspace{2cm} &&\text{\color{mypink}\small Integration by parts formula}\\
 \int xe^x \ dx &= xe^x - \int e^x  dx &&\text{\color{mypink}\small Substitute.}\\
 &= xe^x - e^x + C. &&\text{\color{mypink}\small Integrate.}
\end{align*}
\bigskip
\noindent To check this, differentiate $xe^x - e^x + C$ to see that you obtain the original integrand.

\sffamily
\noindent \color{myblue} \large \textbf{EXAMPLE} \color{black} \normalsize \textbf{Integration by Parts}
\rmfamily

\bigskip

\noindent Find $\displaystyle \int x^2 \ln x \ dx$.

\bigskip

\sffamily
\noindent \color{myblue} \textbf{SOLUTION} \color{black} \rmfamily \hspace{.3cm} In this case, $x^2$ is more easily integrated that $ln x$. So, you should let $dv = x^2 dx$.

\begin{alignat*}{3}
dv &= x^2 \hspace{.1cm}dx \hspace{.3cm}&&\color{mypink}\Rightarrow\color{black}& v &= \int x^2 \ dx = \frac{x^3}{3}\\
u &= \ln x &&\color{mypink}\Rightarrow\color{black} & \hspace{.3cm}du &= \frac{1}{x} dx
\end{alignat*}

\noindent Integration by parts produces
\begin{alignat*}{2}
\int u \ dv &= uv - \int v \ du \ \hspace{2cm} &&\text{\color{mypink}\small Integration by parts formula}\\
 \int x^2 \ln \ x \ dx &= \frac{x^3}{3} \ln \ x - \int \left(\frac{x^3}{3}\right) \left(\frac{1}{x}\right) dx \hspace{2cm} &&\text{\color{mypink}\small Substitute.}\\
 &= \frac{x^3}{3} \ln \ x - \frac{1}{3} \int x^2 \ dx &&\text{\color{mypink}\small Simplify.}\\
 &= \frac{x^3}{3} \ln \ x - \frac{x^3}{9} + C. &&\text{\color{mypink}\small Integrate.}
\end{alignat*}
You can check this result by differentiating.

$$ \frac{d}{dx} \left[\frac{x^3}{3} \ \ln \ x \ - \frac{x^3}{9} \right] = \frac{x^3}{3} \left(\frac{1}{x} \right) + (\ln \ x)(x^2) - \frac{x^2}{3} = x^2 \ \ln \ x  \hspace{4cm} \color{myblue} \filledmedsquare $$

\vspace{1cm}

One surprising application of integration by parts involves integrands consisting of single terms, such as $\int \ln x \ dx$ or $ \int \arcsin x \ dx$. In these cases, try letting $dv = dx$, as shown in the next example.

\bigskip

\sffamily
\noindent \color{myblue} \large \textbf{EXAMPLE} \color{black} \normalsize \textbf{An Integrand with a Single Term}
\rmfamily

\bigskip

\noindent Evaluate $ \displaystyle \int_0^1 \arcsin x \ dx$.

\bigskip

\sffamily
\noindent \color{myblue} \textbf{SOLUTION} \color{black} \rmfamily \hspace{.3cm} Let $dv = dx$.

\begin{alignat*}{3}
dv &= dx       &  \hspace{.3cm} &\color{mypink} \Rightarrow \color{black}   & \hspace{.5cm}  v&= \int dx \ = \ x \\
u &= \arcsin x &   &\color{mypink} \Rightarrow \color{black}   &   du&= \frac{1}{\sqrt{1-x^2}} dx \\
\end{alignat*}

\noindent Integration by parts now produces
\begin{alignat*}{2}
\int u \ dv &= uv - \int v \ du \ \hspace{2cm} &&\text{\color{mypink}\small Integration by parts formula}\\
 \int \arcsin \ x \ dx &= x \arcsin x - \int \frac{x}{\sqrt{1-x^2}} \ dx \hspace{2cm} &&\text{\color{mypink}\small Substitute.}\\
 &= x \arcsin x + \frac{1}{2} \int (1 - x^2)^{-1/2} (-2x) \ dx \hspace{1.5cm} &&\text{\color{mypink}\small Rewrite.}\\
 &= x \arcsin x + \sqrt{1 - x^2} + C. &&\text{\color{mypink}\small Integrate.}
\end{alignat*}

\noindent Using this antiderivative, you can evaluate the definite integral as follows.
\begin{align*}
\int_0^1 \arcsin x \ dx &= \bigg[ \ x \arcsin x + \sqrt{1 - x^2} \ \bigg]_0^1 \\
&= \frac{\pi}{2} - 1 \\
&\approx 0.571
\end{align*}

\noindent The area represented by this definite integral is shown in Figure 8.2. \hspace{4cm} $\color{myblue} \filledmedsquare$

\vspace{1cm}

Some integrals require repeated use of the integration by parts formula. 

\bigskip

\sffamily
\noindent \color{myblue} \large \textbf{EXAMPLE} \color{black} \normalsize \textbf{Repeated Use of Integration by Parts}
\rmfamily

\bigskip

\noindent Find $ \displaystyle \int x^2 \sin x \ dx$.

\bigskip

\sffamily
\noindent \color{myblue} \textbf{SOLUTION} \color{black} \rmfamily \hspace{.3cm} The factors $x^2$ and $\sin x$ are equally easy to integrate. However, the derivative of $x^2$ becomes simpler, whereas the derivative of $sin x$ does not. So, you should let $u = x^2$.

\begin{alignat*}{3}
dv &= \sin x \ dx \quad &\color{mypink}\Rightarrow\color{black} \quad  v &= \int \sin x \ dx = -\cos x \\
u &= x^2 &\color{mypink}\Rightarrow\color{black} \quad  du &= 2x dx
\end{alignat*}

\noindent Now, integration by parts produces

$$ \int x^2 \sin x \ dx = -x^2 \cos x + \int 2x \cos x \ dx. \hspace{2cm} \text{\color{mypink}\small First use of integration by parts} $$

\noindent The first use of integration by parts has succeeded in simplifying the original integral, but the integral on the right still doesn't fit a basic integration rule. To evaluate that integral, you can apply integration by parts again. This time, let $u = 2x$.

\begin{alignat*}{3}
dv &= \cos x \ dx \quad &\color{mypink}\Rightarrow\color{black} \quad v &= \int \cos x \ dx = \sin x \\
u &= 2x \quad &\color{mypink}\Rightarrow\color{black} \quad du &= 2 dx
\end{alignat*}

\noindent Now, integration by parts produces

\begin{align*}
\int 2x \cos x \ dx &= 2x \sin x - \int 2 \sin x \ dx \hspace{2cm} \text{\color{mypink}\small Second use of integration by parts} \\
&= 2x \sin x + 2 \cos x + C.
\end{align*}

\noindent Combining these two results, you can write

$$ \int x^2  \sin \ x \ dx \ = \ -x^2  \cos \ x \ + \ 2x \ \sin \ x \ + \ 2 \ \cos \ x \ + \ C. \hspace{4cm} \color{myblue} \filledmedsquare$$

\vspace{1cm}

When making repeated applications of integration by parts, you need to be careful not to interchange the substitutions in successive applications. For instance, in Example 4, the first substitution was $u = x^2$ and $dv = \sin \ x \ dx$. If, in the second application, you had switched the substitution to $u = \cos$ and $dv = 2x$, you would have obtained

\begin{align*}
\int x^2 \ \sin x \ dx &= -x^2 \ \cos x \ + \int 2x \cos x \ dx \\
&= -x^2 \ \cos x + x^2 \ \cos x + \int x^2 \ \sin x \ dx = \int x^2 \ \sin x \ dx
\end{align*} 

\noindent thereby undoing the previous integration and returning to the \textit{original} integral. When making repeated applications of integration by parts, you should also watch for the appearance of a \textit{constant multiple} of the original integral. For instance, this occurs when you use integration by parts to evaluate $\int e^x \cos 2x dx$, and also occurs in Example 5 on the next page.

The integral in Example 5 is an important one. In Section 8.4 (Example 5), you will see that it is used to fine the arc length of a parabolic segment. 

\bigskip

\noindent \color{myblue} \large \textbf{EXAMPLE} \color{black} \normalsize \textbf{Integration by Parts}
\rmfamily

\bigskip

\noindent Find $\displaystyle \int \sec^3 x \ dx$.
\sffamily

\bigskip

\noindent \color{myblue} \textbf{SOLUTION} \color{black} \rmfamily \hspace{.3cm}The most complicated portion of the integrand that can be easily integrated in $\sec^2 x \ dx$, so you should let $dv = sec^2 x \ dx$ and $u = \sec x$.

\begin{alignat*}{3}
dv &= \sec^2 x \ dx       &  \hspace{.3cm} &\color{mypink} \Rightarrow \color{black}   & \hspace{.5cm}  v&= \int \sec^2 x \ dx \ = \tan x \\
u &= \sec x &   &\color{mypink} \Rightarrow \color{black}   &   du&= \sec x \tan x dx \\
\end{alignat*}
\pagebreak

\noindent Integration by parts produces

\begin{align*}
\int u \ dv &= uv - \int v \ du \hspace{5cm} &&\text{\color{mypink}\small Integration by parts formula}\color{black}\\
\int \sec^3 x \ dx &= \sec x \ \tan x - \int \sec x \ \tan^2 x \ dx  &&\text{\color{mypink}\small Substitute.}\color{black}\\
\int \sec^3 x \ dx &= \sec x \ \tan x - \int \sec x (\sec^2 x - 1) \ dx  &&\text{\color{mypink}\small Trigonometric indenty}\color{black}\\
\int \sec^3 x \ dx &= \sec x \ \tan x - \int \sec^3 x \ + \int \sec x \ dx  &&\text{\color{mypink}\small Rewrite.}\color{black}\\
2 \int \sec^3 x \ dx &= \sec x \ \tan x + \int \sec x \ dx  &&\text{\color{mypink}\small Collect like integrals.}\color{black}\\
2 \int \sec^3 x \ dx &= \sec x \ \tan x + \ln |\sec x + \tan x| + C  &&\text{\color{mypink}\small Integrate.}\color{black}\\
\int \sec^3 x \ dx &= \frac{1}{2} \sec x \ \tan x + \frac{1}{2}\ln |\sec x + \tan x| + C  &&\text{\color{mypink}\small Divide by 2.}\color{black}\\
\end{align*}

As you gain experience in using integration by parts, your skill in determining $u$ and $dv$ will increase. The following summary lists several common integrals with suggestions for the choices of $u$ and $dv$.

\bigskip
\sffamily
\begin{tcolorbox}[colback = beige!75!white,
				  sharp corners = all,
				  colframe = beige!75!white]
\textbf{SUMMARY OF COMMON INTEGRALS USING INTEGRATION BY PARTS}
\rmfamily
%Try to move indent left
\begin{enumerate}
\item[\textbf{1.}] For integrals of the form \\
$$ \int x^n \ e^{ax} \ dx, \quad \int x^n \sin ax \ dx, \quad \text{or} \quad \int x^n \cos ax \ dx $$ \\
let $u = x^n$ and let $dv = e^ax \ dx$, $\sin ax \ dx$, or $ \cos ax \ dx$.

\item[\textbf{2.}] For integrals of the form \\
$$ \int x^n \ln x \ dx, \quad \int x^n \arcsin ax \ dx, \text{or} \quad \int x^n \arctan ax \ dx $$ \\
let $u = \ln x$, $\arcsin ax$, or $\arctan ax$ and let $dv = x^n dx$.

\item[\textbf{3.}] For integrals of the form \\
$$ \int e^{ax} \sin bx \ dx \quad \text{or} \quad \int e^{ax} \cos bx \ dx$$ \\
let $u = \sin bx$ or $\cos bx$ and let $dv = e^{ax} \ dx$.

\end{enumerate}
\end{tcolorbox}

\cite{Calc}

\chapter{Slides}
\includegraphics[scale = .85]{slide1.jpg}\\
\includegraphics[scale = .85]{slide2.jpg}

\newpage
\begin{center}
\includegraphics[scale = .95]{slide3.jpg}\\
\includegraphics[scale = .95]{slide4.jpg}
\end{center}

\newpage
\begin{center}
\includegraphics[scale = .95]{slide5.jpg}\\
\includegraphics[scale = .95]{slide6.jpg}
\end{center}

\newpage
\begin{center}
\includegraphics[scale = .95]{slide7.jpg}\\
\includegraphics[scale = .95]{slide8.jpg}
\end{center}

\newpage
\begin{center}
\includegraphics[scale = .95]{slide9.jpg}\\
\includegraphics[scale = .95]{slide10.jpg}
\end{center}

\cite{IntegrationByParts}

\chapter{Short Mathematical Papers}
\section{Paper 1}
\vspace{1in} \rmfamily
\begin{center}
\Large{\bfseries COUNTEREXAMPLE TO EULER’S CONJECTURE ON SUMS OF LIKE POWERS} \\
\bigskip 
BY L. J. LANDER AND T.R. PARKIN \\
Communicated by J. D. Swift, June 27, 1966
\end{center}
\bigskip
A direct search on the CDC 6600 yielded
$$27^5 + 84^5 + 133^5 = 144^5$$
As the smallest instance in which four fifth powers sum to a fifth power. This is a counterexample to a conjecture by Euler [1] that at least n nth powers are required to sum to an \textit{n}th power, $n>2$.
\bigskip

\begin{center} \textsc{Reference} \end{center}
1.	L. E. Dickson, History of the theory of numbers, Vol. 2, Chelsea, New York, 1952, p.648.\\

\cite{CounterToEuler}

\newpage
\section{Paper 2}
\begin{center}
\vspace{2in}
\Large{\bfseries On a Conjecture of R. J. Simpson About Exact Covering Congruences}\\ \normalsize
\bigskip
\textsc{Doron Zeilberger$^1$}\\
\bigskip
Department of Mathematics, Drexel University, Philadelphia, PA 19104
\end{center}
\bigskip
The following is a counterexample$^2$ to Simpson’s conjecture [\textbf{2}]: D = \{6, 15, 35, 14, 210 (140 times)\}. It was concocted using the elegant and powerful approach of [\textbf{1}].\\

\bigskip

\noindent REFERENCES\\
\noindent 1. Mark A. Berger, Alexander Felzenbaum, and Aviezri S. Fraenkel, New results for covering systems of residue sets, Bulletine (New Series) of the Amer. Math. Soc., 14 (1986) 121-125.

\noindent 2. R. J. Simpson, Disjoint covering systems of congruences, this \textsc{Monthly}, 94 (1987) 865-868.\\

\cite{ConjectureSimpson}


\chapter{Letters}
\begin{flushleft}

Bernadette Hoffman\\
123 Road Lane\\
City, WV 25302\\
\vspace{.25in}

December 5, 2021\\
\vspace{.25in}

President Joe Biden\\
1600 Pennsylvania Avenue NW\\
Washington, DC 20500\\
\vspace{.25in}

Dear Mr. President,\\
\vspace{.25in}

I hope all finds you well in this tumultuous time. I am writing today to remind you that although our country continues to struggle with COVID-19 pandemic, that calculus remains of the upmost importance. I am not sure how much one focuses on mathematics in politics outside of statistical anomalies etc. Which is why I worry that it is not considered an important topic at this time. Mathematics is very important to me, more specifically Calculus. \\
\bigskip
Children in the public school system today are offered calculus classes as part of the normal curriculum. However, I am concerned that what is passed off as calculus at that level is barely pre-calculus as far as the university level is concerned. Children having taken calculus in high school are hardly even introduced to many of the very important fundamental concepts of calculus itself. I feel that this is important because it does not seem to properly prepare them for a college career. Going to college in the first place is a daunting task as it is. I feel therefore that a re-naming is in order for high school students taking calculus. I believe it should be re-labeled “pre-calculus” and that higher level calculus classes should be offered at an A.P. level that more closely resembles the curriculum at a university level. I feel that this will more properly prepare young students for math curriculum in college. \\
\bigskip
This is extremely important for the future of this county. Math is the foundation of many career opportunities. More and more children are discouraged and drop out of math heavy programs due to being under-prepared by their high school education.\\
\vspace{.25in} 

Sincerely,\\ 

Bernadette Hoffman, Concerned Student
\end{flushleft}

\newpage

\begin{flushleft}

Bernadette Hoffman\\
123 Road Lane\\
City, WV 25302\\
\vspace{.25in}

December 5, 2021\\
\vspace{.25in}

Mr. Leonard Euler\\
Kohlenberg 7 \\
4051 Basel, Switzerland\\
\vspace{.25in}

Dear Mr. Euler,\\

I hope that you were aware, while present in your perspective modern day society, of the magnitude of your contributions to the mathematics for ions to come. Currently concepts of yours are in use in modern engineering, computation sciences, astronomy and more. Your name is not easily missed in Calculus, Algebra, or Numerical Analysis. You may be pleased to know that every teacher I have had has made sure to supply the correct pronunciation of your name. Clearly, your impact on mathematics has earned you a prominent place in the minds of many. \\
\bigskip
As I ponder over the greatest of your contributions, I must point out, that in today's society, impact is highly subjective. Because of this, the contribution I find most meaningful (partially due to limits on comprehension) is going to be highly subjective as well. While I know that you've made greater accomplishments, and countless theorems and formulas containing your name, do not equate this to the most useful of your concepts. \\
\bigskip
On that note I will inform you, that my favorite formula of yours, is actually a theorem. A theorem about perfection. Your theorem that states that a number is perfect if and only if it has the form:
$$2^{p-1}(2^p-1).$$
I have discovered that your theorem is that based on a number called a Mersenne Prime number:
$$M_p=2^{p-1}.$$
Thus the Euclid-Euler Theorem is then:
$$Perfect \ number = 2^{p-1}M_p.$$\\
\bigskip
In conclusion, thank you for your immense contributions to math and society in general. It would have been a pleasure to meet you.\\
\vspace{.25in}
Sincerely, \\
\bigskip
Bernadette Hoffman
\end{flushleft}

\cite{Letters}

\newpage


\printbibliography[heading = bibintoc, title={References}]

\end{document}